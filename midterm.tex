\documentclass[12pt, letterpaper, oneside]{article}
\usepackage{xeCJK}
\usepackage{amsmath}
\usepackage{enumitem} % \enumerate styles

\setCJKmainfont{ukai.ttc}
\setCJKmonofont{ukai.ttc}

\newcommand
{\bookProblem}
[1]
{\subsection{第{#1}頁}}

\title{\textbf{個體經濟 期中考範圍}}
\author{北科大 資工系 106}
\date{\today}

\begin{document}

\maketitle

\section{選擇題(40份)}

\bookProblem{3}
傳統個體經濟學的分析架構包括:
\begin{itemize}
    \item 消費者行為
    \item 產生者行為
    \item 市場
    \item 市場失靈與一般均衡
\end{itemize}

\bookProblem{18}
下列敘述何者屬於個體經濟學的範疇?
\begin{enumerate}[label=(\Alph*)]
    \item 膨風嫂大學畢業後,決定到中國大陸工作。
    \item 印尼財長預估 2017年4月 起的年度經濟成長率,可能由前一年度的 7.2\% 提升至 8.2\%,且工業產生年增幅度創下自 1944 年以來最大規模的 17.6\%。
    \item 全球經濟復甦加上日圓疲軟帶動對汽車與半導體的需求, 2016年7月出口強勁成長近 30\%。
    \item 2017年 4K2K 電視產能激增,售價恐下跌 40\%。
\end{enumerate}

\bookProblem{25}
除了產品本身價格以外,其它因素也會影響需求量。這些因素包括:
\begin{itemize}
    \item 所得
    \item 其它相關產品價格
    \item 偏好
    \item 預期物價
    \item 其它
\end{itemize}

\bookProblem{36}
均衡價格與數量的變動
\begin{itemize}
    \item 需求增加+供給增加 = 價格不定 與 數量增加
    \item 需求增加+供給減少 = 價格上升 與 數量不定
    \item 需求減少+供給增加 = 價格下跌 與 數量不定
    \item 需求減少+供給減少 = 價格不定 與 數量減少
\end{itemize}

\bookProblem{67}
無異曲線的特性
\begin{itemize}
    \item 無異曲線 斜率為負
    \item 無異曲線 不會相交
    \item 無異曲線 凸向原點
    \item 無異曲線 不厚
\end{itemize}

\bookProblem{133}
網路外部性
\begin{itemize}
    \item 隨波逐流效果 bandwagon effect
    \item 鄉願效果 snob effect
    \item 韋伯論效果 Veblen effect
\end{itemize}

\bookProblem{154}
假設阿亮只消費三種商品:大麥克,署條與可樂,這三種商品可否都是正常財?這三種商品可否都是奢侈品?

\bookProblem{173}
等產量線的特性
\begin{itemize}
    \item 等產量線 愈往右上方,代表產量愈大
    \item 等產量線 斜率為負
    \item 等產量線 不會相交
    \item 等產量線 凸向原點
\end{itemize}

\bookProblem{187}
單一投入的生產函數稱為總產量函數,總產量函數有三個階段:
\begin{enumerate}
    \item 邊際報酬遞增
    \item 邊際報酬遞減
    \item 總報酬遞減
\end{enumerate}

\bookProblem{236}
完全競爭的特性
\begin{itemize}
    \item 價格接受者 price takers
    \item 廠商生產 同質產品 homogeneous product
    \item 買賣雙方對市場價格具完全資訊
    \item 廠商可以 自由進出市場 freely entry and exit
\end{itemize}

\section{問答題(30份)}

\bookProblem{58}
假設蕭敬勝的 DVD 紀念專輯供給曲線可以下列方程式表示:
\begin{equation*}
Q^s = -9 + 5P -2.5R
\end{equation*}
其中,$Q^s = $專輯供給量,$P = $專輯價格,$R = $製作 DVD 的原料。
\begin{enumerate}[label=(\alph*)]
    \item 請問在 R = 2 與 P = 12 時,供給量是多少?請問在 R = 4 與 P = 12 時,供給量又是多少?
    \item 請問供給方程式是否符合供給法則?
\end{enumerate}

\bookProblem{229}
阿亮在網路上開創重型機車的買賣。有興趣的人會在網路上下訂單,阿亮則將訂單蒐集,然後將重型機車送到客戶的手中。為了要經營網購,每個月阿亮支付給網際網路公司 50,000元來使用網頁以及維護網頁的費用。為了送貨,阿亮擁有一輛貨車,每個月支付 30,000元的分期貸款和 10,000元的保險費。每一筆訂單的平均汽油費用是 500元。若阿亮不開創網購事業,他可以在補習班教書,每小時鐘點費 3,000元。
\begin{enumerate}[label=(\alph*)]
    \item 阿亮的外顯成本與隱含成本為何?經濟成本與會計成本為何?
    \item 若阿亮決定結束網購事業,此決策的非沉沒成本為何?哪些成本是沉沒成本?
\end{enumerate}

\subsection{心得}
\begin{enumerate}
    \item 中美貿易戰 或 美味代價 Food, Inc. 的心得與感想?
    \item 期中考之前對這門課的心得與感想?
\end{enumerate}


\section{計算題(30份)}

\bookProblem{57}
百事可樂與可口可樂一直是軟性飲料市場的兩大龍頭,如果可口可樂的需求函數為 $Q_c = 26 - 2P_c + I$,其中 $Q_c = $可口可樂的需求量(單位:打),$P_c = $可口可樂價格(單位:元),$I = $消費者所的(單位:千元)。
\begin{enumerate}[label=(\alph*)]
    \item 當可口可樂價格等於 12元,和消費者所得等於 20元時,可口可樂需求量是多少?
    \item 當可口可樂價格等於 15元,和消費者所得等於 20元時,可口可樂需求量是多少?
    \item 可口可樂是正常財或劣等財?
    \item 若需求函數改成:$Q_c = 26 - 2P_c + 2P_p + I$,若可口可樂價格($P_c$)等於 12元,百事可樂價格($P_p$)等於 10元,和消費者所得($I$)等於 20元,則可口可樂需求量是多少?可口可樂與百事可樂為替代品或互補品?
\end{enumerate}

\bookProblem{154}
假設台北市的租屋有兩個市場區隔:商務辦公與一般住宅。商務辦公的市場需求函數為 $Q_1 = 40 - 0.25P$;而一般住宅需求函數為 $Q_2 = 120 - 2P$。其中 $Q_1$ 和 $Q_2$ 是商務辦公與一般住宅的需求量,$P$ 是每坪價格。
\begin{enumerate}[label=(\alph*)]
    \item 請填滿下列 Table \ref{table:data} 不同價格的租屋需求。
    \item 請劃出個別市場與整個市場的需求曲線,並寫出市場需求函數。
\end{enumerate}
\begin{table}[h!]
    \centering
    \begin{tabular}{ |c|c|c|c| }
        \hline
        價格 (P) & $Q_1$ & $Q_2$ & 市場需求 \\
        \hline
        100 &&& \\
        80  &&& \\
        60  &&& \\
        40  &&& \\
        20  &&& \\
        0   &&& \\ [1ex]
        \hline
    \end{tabular}
    \caption{填滿數字就可以了}
    \label{table:data}
\end{table}

\end{document}
